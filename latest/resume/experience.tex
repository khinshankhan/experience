%-------------------------------------------------------------------------------
%	SECTION TITLE
%-------------------------------------------------------------------------------
\cvsection{Work Experience}


%-------------------------------------------------------------------------------
%	CONTENT
%-------------------------------------------------------------------------------
\begin{cventries}

%---------------------------------------------------------

  \nrolecventrynoitems
    {Kiip, PBC} % Organization
    {Brooklyn, New York} % Location
    {Senior Software Engineer} % Job title
    {Dec. 2022-Present} % Date(s)
    {Founding Software Engineer} % Job title
    {Dec. 2021-Dec. 2022} % Date(s)
    {} % Job title
    {} % Date(s)

  \cvcustombodydescription
    {\myhy{https://www.kiip.io}{\textbf{kiip.io}} and \myhy{https://www.kiip.co}{\textbf{kiip.co}} (\emph{AWS Lambda, S3, DynamoDB, Chime, GraphQL, Next.js, MobX, Javascript, Typescript})}
    {
      \begin{cvtightprose} % Description(s) of tasks/responsibilities
        {Kiip allows people to store digital copies of vital documents and share them in interactive data rooms when applying for services offered by the city or other public benefits and service providers, easing any potential times of crisis or extenuating circumstances.}
      \end{cvtightprose}
    }
    {
      \begin{cvitems} % Description(s) of tasks/responsibilities
        \item {Setup engineering foundations ranging from separate amplify environments to DNS management to GitHub Actions to brown bags}
        \item {Designed scalable and resuable \textbf{React components} ranging from skeletal layout to chat messaging to meet \textbf{WCAG accessibility standards}}
        \item {Created a service for users to securely upload documents with `zero knowledge' of its contents until a user with proper rights accesses it}
        \item {Established various \textbf{authorization models} for individual users to add assistants to work on their behalf to interact with representatives of organizations as well as varying levels of representatives of an organization}
        \item {Lead demos for stakeholders and investors, distilling technical information to intuitive formats such as diagrams, analogies, simplified personas}
      \end{cvitems}
    }

%---------------------------------------------------------
  \nrolecventrynoitems
    {Two Bulls} % Organization
    {Brooklyn, New York} % Location
    {Software Engineer} % Job title
    {Jan. 2020-Jan. 2022} % Date(s)
    {} % Job title
    {} % Date(s)
    {} % Job title
    {} % Date(s)

  \cvcustombodydescription
      {\textbf{Port Management Tool} (\emph{AWS Lambda, S3, DynamoDB, PostgreSQL, Golang, Typescript, React, Leaflet, Material UI, Node.js})}
      {
      \begin{cvtightprose} % Description(s) of tasks/responsibilities
        {Working with \myhy{https://seacorholdings.com}{Seacor Holdings}, several of their ports now have ship detection and corresponding port events handled and searchable automatically via this tool and added to billing for \myhy{https://www.helmoperations.com}{Helm Operations} orders for cross-examination and detailed view of ships' paths.}
      \end{cvtightprose}
      }
      {
      \begin{cvitems} % Description(s) of tasks/responsibilities
        \item {Led cooperative agreements with data providers to create new APIs and selectively clean noise, handling \textbf{1,000,000+ daily} AIS messages, as well as agreeable \textbf{TTL policies} for caching vessel particulars with custom round robin scheduled refetching data}
        \item {Implemented \textbf{scalable and resilient microservice architecture} (lambdas, databases (\emph{DynamoDB \rightarrow PostgreSQL}), S3, VPSs)}
        \item {Spearheaded aspects of the `complex event processing' algorithm for port events and port calls with pipelines to smoothly incorporate new categorizations of vessels and geofencing which will affect port events used for billing}
        \item {Worked on client product for port call search and display summary with an interactive vessel path visualization}
        \item {Initiated an internal algorithm visualization playback, playing around \textbf{100,000 records as a smooth animation} at a time}
      \end{cvitems}
    }

  \cvcustombodydescription
      {\textbf{Mailchimp Firebase Integration} (\emph{Typescript, Firebase, GCP, Mailchimp})}
      {
      \begin{cvtightprose} % Description(s) of tasks/responsibilities
        {Syncs user data with a Mailchimp audience for sending personalized email marketing campaigns. It was teased at Google I/O 2021 and presented at Google I/O 2022. The extension was \myhy{https://extensions.dev/extensions/mailchimp/mailchimp-firebase-sync}{open sourced} by \myhy{https://mailchimp.com}{Mailchimp}.}
      \end{cvtightprose}
      }
      {
      \begin{cvitems} % Description(s) of tasks/responsibilities
        \item {Headed initial development for \textbf{alpha Firebase extension} and tests (unit, integration, E2E) with \textbf{GitHub Actions}}
        \item {Developed Firebase extension to bidirectionally synchronize with Mailchimp tags, audiences, and events for personalized marketing}
      \end{cvitems}
    }

%---------------------------------------------------------
\end{cventries}
